\documentclass[11pt]{article}
\usepackage[margin=1in]{geometry}
\usepackage{amsmath, amssymb, amsthm}
\usepackage{microtype}
\usepackage[T1]{fontenc}
\usepackage[utf8]{inputenc}
\usepackage{hyperref}
\usepackage{times}

\title{A Boundary Product--Certificate Proof of the Riemann Hypothesis}
\author{Jonathan Washburn}
\date{\today}

\begin{document}
\maketitle

\begin{abstract}
High-level narrative of the product-certificate route: normalize by $\zeta$, control boundary energy via VK zero density, apply CR/Green pairing, neutralize outer contributions, prove a boundary wedge, transport it to the interior, pinch at hypothetical off-line zeros. Emphasize unconditional proof, explicit constants, and Lean formalization.
\end{abstract}

\section{Introduction}

The Riemann Hypothesis (RH) sits at the crossroads of analytic number theory, complex analysis, and harmonic analysis. Its resolution has inspired myriad approaches, ranging from the classical Hadamard--de la Vallée Poussin method, through Selberg-style trace formulae, to modern random matrix heuristics. The present work contributes a new, single-route proof architecture built around a boundary product--certificate. Instead of juggling multiple partial arguments, we design a product normalization tailored to $\zeta$, neutralize the prime-diagonal operator via a Hilbert--Schmidt determinant, and drive everything through a carefully engineered boundary wedge.

The main novelty is the synthesis of Ford/Korobov zero-density technology with a concrete CR/Green pairing on Whitney tents. The Ford/Korobov engine provides unconditional Victorian-Korobov (VK) zero-free regions and density bounds with explicit constants; these feed a Carleson-energy budget that, after subtracting the Blaschke contributions, matches the window geometry. CR-calculus supplies the quantitative link between boundary phase and interior energy. Once the boundary wedge is locked with explicit constants, Poisson transport and a Schur pinch dismiss any hypothetical off-critical zero.

Section~2 recalls the product certificate and the associated normalization; Section~3 develops the Ford/Korobov analytic number theory input, culminating in VK zero density with fully explicit exponents; Section~4 converts zero counts to Carleson energy on Whitney boxes; Section~5 proves the phase-velocity identity and the CR/Green pairing; Section~6 establishes the window/wedge inequalities; Section~7 globalizes the wedge and executes the final pinch; Section~8 describes the computer verification and Lean implementation details. Appendices provide detailed constant tables and auxiliary lemmas.

Large portions of this proof are mechanized in Lean. The repository encodes the product-certificate structure, the zero-density derivations, the CR/Green calculus, and the wedge logic. Automation plays two roles: it ensures every algebraic or measure-theoretic reduction is watertight, and it records the exact constants needed to close the inequalities. Whenever a classical argument relies on ``bounded error'' heuristics, the Lean proof supplies the concrete bound and tracks dependencies between files. The remaining manual sections are limited to the Ford/Korobov constants yet to be formalized and the final numeric audit, both scheduled for inclusion in the next release.

\section{Product Certificate and Normalization}

\subsection{Prime-diagonal operator and modified determinant}

Let \(A(s) : \ell^2(\mathcal P) \to \ell^2(\mathcal P)\) be the prime-diagonal operator defined by
\[
  A(s) e_p = p^{-s} e_p, \qquad s = \sigma + it, \quad \sigma > \tfrac12.
\]
The Hilbert--Schmidt norm satisfies \(\|A(s)\|_{\mathrm{HS}}^2 = \sum_{p} p^{-2\sigma} < \infty\), so \(A(s)\) is trace class on any half-plane \(\sigma > \tfrac12\). We work with the two-modified determinant
\[
  \det\nolimits_2 (I - A(s)) := \det\big( (I - A(s)) \exp(A(s) + \tfrac12 A(s)^2) \big),
\]
which converges absolutely and captures the \(k \ge 2\) prime powers in the Euler product. This choice is tailored to the completed zeta function and admits uniform control on vertical strips.

\subsection{Outer normalization and the function \(J(s)\)}

To offset the archimedean and \(k=1\) contributions, we divide by a carefully chosen outer function \(\mathcal O_\zeta(s)\) (constructed via a logarithmic Hilbert transform on the boundary). The key object is
\[
  J(s) := \frac{\det\nolimits_2(I - A(s))}{\mathcal O_\zeta(s) \, \zeta(s)}.
\]
Because \(\det\nolimits_2(I - A)\) and \(\zeta(s)\) share the same prime poles/zeros after neutralization, \(J(s)\) is holomorphic on \(\Omega = \{\Re s > \tfrac12\}\), with unimodular boundary values except at zeros of \(\xi(s)\). The outer factor enforces \(|J(1/2 + it)| = 1\) almost everywhere, positioning \(J\) squarely in the boundary product setting.

\subsection{Cayley transform to the Schur function}

Define
\[
  \Theta(s) := \frac{2J(s) - 1}{2J(s) + 1}.
\]
When \(2J\) has nonnegative real part, \(\Theta\) is a Schur function on \(\Omega\). The poles/zeros of \(\Theta\) correspond to those of \(\xi(s)\), while normalization at infinity gives \(\Theta(\sigma + it) \to -1\) as \(\sigma \to +\infty\). The Schur framework allows us to transport boundary inequalities to the interior via the Herglotz representation.

\subsection{Boundary wedge, Poisson transport, and Schur pinch}

The boundary wedge condition (P+) asserts that
\[
  -\frac{\pi}{2} + \delta \le \arg J\big(\tfrac12 + it\big) \le \frac{\pi}{2} - \delta
\]
for all \(t\) outside a negligible set and some fixed \(\delta > 0\). We prove P+ by pairing boundary phase with Whitney windows: CR/Green identities bound the phase variation by the Carleson energy supplied from VK zero density, while Poisson plateau estimates give a matching lower bound proportional to the zero count. Once P+ holds, the boundary Poisson integral shows \(\Re(2J(s)) \ge 0\) in \(\Omega\), so \(\Theta\) is Schur throughout the domain. Finally, the Schur pinch contradicts the existence of an off-critical zero: if \(\xi(\rho) = 0\) with \(\Re \rho > \tfrac12\), then \(\Theta(\rho) = 1\) but \(\Theta\) tends to \(-1\) at infinity; by the maximum modulus principle, this is impossible, and thus RH follows.

\section{Analytic Number Theory Engine (Ford/Korobov)}

\subsection{Ford exponential sum hypothesis and constants}

Write
\[
  S(X,t) := \sum_{n\le X} n^{-it}.
\]
The Ford/Korobov technology yields, for \(X,t \ge 2\),
\begin{equation}\label{eq:ford}
  |S(X,t)| \;\le\; A \, X^{1-\theta}\, t^{\theta} \;+\; B\, X^{1/2},
\end{equation}
with explicit constants \(A,B>0\) and an exponent \(\theta \in (0,1)\) determined by the underlying exponent pair (classically \(\theta\) may be taken near \(1/6\)). These constants propagate through Abel summation, the approximate functional equation, and Littlewood-type inequalities to control \(\zeta\) on the critical strip.

\subsection{Abel summation and Dirichlet polynomial bounds}

Let \(S_\sigma(X,t) := \sum_{n\le X} n^{-\sigma-it}\) for \(1/2 \le \sigma \le 1\). Abel summation gives the exact identity
\begin{equation}\label{eq:abel}
  S_\sigma(X,t) \;=\; X^{-\sigma} S(X,t) \;+\; \sigma \int_{1}^{X} u^{-\sigma-1}\, S(u,t)\, du.
\end{equation}
A key technical point in our formal development is a \emph{positive-index partial summation} that aligns the boundary term with the Ford bound at the same cutoff (thereby avoiding a loss when \(\sigma\) approaches \(1\)). Combining \eqref{eq:abel} with \eqref{eq:ford} yields
\begin{equation}\label{eq:dirichlet-poly}
  \Bigl|\sum_{n\le X} n^{-\sigma-it}\Bigr| \;\ll\;
  A\Bigl( X^{1-\theta-\sigma} + \frac{\sigma}{\sigma-(1-\theta)} \Bigr) t^\theta
  \;+\;
  B\Bigl( X^{1/2-\sigma} + \frac{\sigma}{\sigma-\tfrac12} \Bigr),
\end{equation}
with the natural interpretation that the denominators are present only away from the endpoints. In particular, for fixed \(\sigma\in(1/2,1)\) this realizes the expected two-term Dirichlet polynomial bound.

\subsection{Approximate functional equation and log\(|\zeta|\) bounds}

In the critical strip, the approximate functional equation writes \(\zeta(\sigma+it)\) as the sum of two Dirichlet polynomials of length \(\asymp t^{1-\sigma}\), plus a small error. With \eqref{eq:dirichlet-poly} in hand, one obtains quantitative bounds of the form
\[
  \log |\zeta(\sigma+it)| \;\le\; C \, (\log t)^{2/3} (\log\log t)^{1/3}
\]
uniformly for \(3/4 \le \sigma \le 1\), \(t \ge 3\), in line with the classical VK growth theory. These inequalities feed into both the zero-free region and averaged \(\log^+|\zeta|\) estimates.

\subsection{Littlewood--Jensen rectangle, VK integral bounds, and zero-free region}

Applying Jensen's formula to rectangles \([\sigma_0,\sigma_1]\times [0,T]\) and bounding the boundary integrals by the previous step yields the Vinogradov--Korobov (VK) \emph{integral bound}
\[
  \int_{0}^{T} \log^+|\zeta(\sigma+it)|\, dt \;\ll\; T^{\,1-\kappa(\sigma)} (\log T)^{B},
\qquad
  \frac12 \le \sigma < 1,
\]
for an explicit exponent function \(\kappa(\sigma)\) and some \(B>0\). In parallel, the standard Hadamard--de la Vallée Poussin method combined with VK inputs yields the VK zero-free region
\[
  \zeta(s) \ne 0 \quad \text{for}\quad \Re s \ge 1 - \frac{c}{(\log |t|)^{2/3}} \quad (|t|\ge 3),
\]
with an explicit \(c>0\).

\subsection{Zero density with explicit \(\kappa(\sigma)\)}

Let \(N(\sigma,T)\) denote the number of nontrivial zeros \(\rho=\beta+i\gamma\) with \(\beta\ge \sigma\) and \(0<\gamma\le T\). The VK theory implies, for \(3/4 \le \sigma < 1\) and \(T\) sufficiently large,
\begin{equation}\label{eq:vk-density}
  N(\sigma,T) \;\le\; C_{\mathrm{VK}} \, T^{\,1-\kappa(\sigma)} (\log T)^{B_{\mathrm{VK}}}.
\end{equation}
In our packaging we adopt the explicit choice
\[
  \kappa(\sigma) \;=\; \frac{3(\sigma-\tfrac12)}{2-\sigma},
\]
which grants rapid decay as \(\sigma \to 1\). The constants \(C_{\mathrm{VK}}, B_{\mathrm{VK}}\) are recorded explicitly for downstream use.

\subsection{Packaging constants in \texttt{VKStandalone}}

For modularity we collect the engine parameters in a thin layer:
\begin{itemize}
  \item \emph{Whitney parameters:} aperture \(\alpha\in[1,2]\) and scale \(c\in(0,1]\). Our locked choice is \(\alpha=\tfrac32\), \(c=\tfrac{1}{10}\).
  \item \emph{VK pair:} a concrete \((C_{\mathrm{VK}}, B_{\mathrm{VK}})\), with a working locked pair used to drive the wedge (e.g. \((10^3,5)\) in the development version).
  \item \emph{Annular coefficients:} far-field coefficients \((a_1,a_2)\) aggregating zero counts over Whitney annuli into a Carleson budget; these enter the assembled constant \(K_{\xi,\mathrm{paper}}\).
\end{itemize}
This separation allows the boundary/wedge analysis to depend only on the abstract schema \eqref{eq:vk-density} and a small set of geometric inputs (Whitney geometry, Poisson-balayage constants), while the analytic number theory engine can be refined independently as sharper exponential sum bounds or density results become available.

\section{Zero Density to Carleson Energy}

\subsection{Whitney intervals, annular counts, and weighted sums}

Let \(I \subset \mathbb R\) be a Whitney interval on the boundary line \(\Re s = \tfrac12\), with length \(|I| = L\) and center \(t_0\). For \(k \ge 0\), define the \(k\)-th Whitney annulus above \(I\) by
\[
  A_k(I) \;:=\; \bigl\{\, s = \tfrac12 + \sigma + i t \,:\, t \in I,\; 2^k L < \sigma \le 2^{k+1} L \,\bigr\}.
\]
Let \(\nu_k(I)\) be the number of zeros of \(\zeta\) in \(A_k(I)\) (counted with multiplicity). The weighted annular sum we use is
\[
  \phi_\nu(k) \;:=\; 4^{-k} \, \nu_k(I),
\]
which reflects the inverse-square decay of the Poisson kernel at height \(\sigma \simeq 2^k L\). The central quantitative task is to bound the partial sums
\[
  \sum_{k=0}^{K} \phi_\nu(k)
  \;\;=\;\; \sum_{k=0}^{K} 4^{-k}\, \nu_k(I)
\]
uniformly in \(I\) and \(K\).

\subsection{Total energy on Whitney boxes}

Let \(Q(I) := I \times (0, L]\) be the Whitney box above \(I\). Write \(U = \Re \log \xi\) for the real part of the log of the completed zeta. The Dirichlet energy of \(U\) on \(Q(I)\) is controlled by a sum of annular contributions. Each zero \(\rho\) contributes like \(\int |\nabla \log|s-\rho||^2 \sigma\, d\sigma\, dt \asymp (\text{height})^{-1}\) per unit area; aggregating over annuli produces the geometric weight \(4^{-k}\). Thus, modulo standard geometric constants,
\[
  \iint_{Q(I)} |\nabla U|^2 \,\sigma\, d\sigma\, dt
  \;\lesssim\; \sum_{k\ge 0} \phi_\nu(k) \;+\; \text{(prime/near-field budgets)}.
\]
The VK bounds furnish control of \(\nu_k(I)\); the dyadic decay \(4^{-k}\) beats the slow growth coming from \((2^k L)\)-dependence, and the Whitney calibration \(L \simeq c / \log t_0\) suppresses residual \(\log\)-factors.

\begin{lemma}[Annular energy bound]\label{lem:annular-energy}
Let \(I\) be a Whitney interval of length \(L\) with tent \(Q(I)\), and let \(U_{\mathrm{zeros}}\) denote the contribution to \(\Re\log\xi\) coming from zeros of \(\zeta\). For \(k\ge 0\) let \(\nu_k(I)\) be the number of zeros in the annulus \(A_k(I)\) defined above. Then there exists a constant \(C_{\mathrm{ann}}\) depending only on the Whitney aperture such that
\[
  \iint_{Q(I)} \bigl|\nabla U_{\mathrm{zeros}}\bigr|^2 \,\sigma\, d\sigma\, dt
  \;\le\; C_{\mathrm{ann}}\, |I| \sum_{k\ge 0} 4^{-k}\, \nu_k(I).
\]
\end{lemma}

\begin{proof}
For a single zero \(\rho = \beta+i\gamma\) the neutralized potential is \(\log|s-\rho|\), whose gradient has size \(|s-\rho|^{-1}\). If \(\rho \in A_k(I)\), then every point in \(Q(I)\) lies at Euclidean distance \(\asymp 2^k L\) from \(\rho\), and the vertical coordinate \(\sigma\) on \(Q(I)\) satisfies \(0<\sigma\le L\). In polar coordinates centered at \(\rho\) one obtains
\[
  \iint_{Q(I)} |\nabla \log|s-\rho||^2 \,\sigma\, d\sigma\, dt
  \;\ll\; |I|\, \frac{L}{(2^k L)^2}
  \;=\; c_\alpha\, |I|\, 4^{-k},
\]
with \(c_\alpha\) determined by the aperture. Summing over the \(\nu_k(I)\) zeros in \(A_k(I)\) and over \(k\ge 0\) gives the stated bound with \(C_{\mathrm{ann}} = c_\alpha\).
\end{proof}

\begin{proposition}[VK annular count]\label{prop:vk-annular-count}
Assume the Vinogradov--Korobov density hypothesis \eqref{eq:vk-density}. Let \(I\) be a Whitney interval centered at \(t_0\) with length \(L\) satisfying \(t_0 - L \ge T_0\) (so that the VK bound applies along the entire window). For \(k\ge 0\) set \(\sigma_k := \tfrac12 + 2^k L\). Then there exists a constant \(C_{\mathrm{VK,ann}}\) depending only on \(C_{\mathrm{VK}}\), \(B_{\mathrm{VK}}\), and the Whitney aperture such that
\[
  \nu_k(I)
  \;\le\;
  C_{\mathrm{VK,ann}}\, L\, t_0^{-\kappa(\sigma_k)} (\log t_0)^{B_{\mathrm{VK}}}
  \qquad (k \ge 0).
\]
\end{proposition}

\begin{proof}
All zeros counted by \(\nu_k(I)\) satisfy \(\beta \ge \sigma_k\) and \(|\gamma - t_0| \le L\). Writing \(T_- := t_0 - L\) and \(T_+ := t_0 + L\), we have
\[
  \nu_k(I) \;\le\; N(\sigma_k, T_+) - N(\sigma_k, T_-).
\]
By \eqref{eq:vk-density},
\[
  N(\sigma_k, T_{\pm})
  \;\le\;
  C_{\mathrm{VK}}\, T_{\pm}^{\,1-\kappa(\sigma_k)} (\log T_{\pm})^{B_{\mathrm{VK}}}.
\]
Consider the function \(F(T) := T^{1-\kappa(\sigma_k)} (\log T)^{B_{\mathrm{VK}}}\) for \(T \ge T_0\). Its derivative satisfies
\[
  F'(T)
  = T^{-\kappa(\sigma_k)} (\log T)^{B_{\mathrm{VK}}}
    \Bigl( 1-\kappa(\sigma_k) + \frac{B_{\mathrm{VK}}}{\log T} \Bigr)
  \;\le\; C_1\, T^{-\kappa(\sigma_k)} (\log T)^{B_{\mathrm{VK}}},
\]
where \(C_1\) depends only on \(B_{\mathrm{VK}}\) and the fact that \(\kappa(\sigma_k) \in (0,1)\) for the range of \(\sigma_k\) under consideration. By the mean value theorem there exists \(\xi \in [T_-, T_+]\) with
\[
  |F(T_+) - F(T_-)|
  \;\le\; C_1\, (T_+ - T_-)\, \xi^{-\kappa(\sigma_k)} (\log \xi)^{B_{\mathrm{VK}}}
  \;\le\; 2 C_1\, L\, t_0^{-\kappa(\sigma_k)} (\log t_0)^{B_{\mathrm{VK}}},
\]
since \(\xi \asymp t_0\) and \(T_+ - T_- = 2L\). The same argument applied to \((\log T)^{B_{\mathrm{VK}}}\) shows that \(F(T_-)\) varies by at most \(C_2 L t_0^{-\kappa(\sigma_k)} (\log t_0)^{B_{\mathrm{VK}}-1}\) when \(\log T\) is perturbed, so the total difference \(F(T_+) - F(T_-)\) is bounded by \(C_3 L t_0^{-\kappa(\sigma_k)} (\log t_0)^{B_{\mathrm{VK}}}\). Combining these estimates yields
\[
  \nu_k(I)
  \;\le\;
  C_{\mathrm{VK}}\, |F(T_+) - F(T_-)|
  \;\le\;
  C_{\mathrm{VK,ann}}\, L\, t_0^{-\kappa(\sigma_k)} (\log t_0)^{B_{\mathrm{VK}}},
\]
with \(C_{\mathrm{VK,ann}} := C_{\mathrm{VK}} C_3\).
\end{proof}

\subsection{The \texorpdfstring{\(VKWeightedSumHypothesis\)}{VKWeightedSumHypothesis}}

Applying Proposition~\ref{prop:vk-annular-count} with \(L = c/\log t_0\) and the explicit choice \(\kappa(\sigma) = \frac{3(\sigma-\tfrac12)}{2-\sigma}\) shows that \(\kappa(\sigma_k) \ge c_*\cdot 2^k L \log t_0\) for some absolute \(c_*>0\). Consequently
\[
  \nu_k(I)
  \;\le\;
  C_{\mathrm{VK,ann}}\, L\, (\log t_0)^{B_{\mathrm{VK}}}
  \exp\!\bigl(-c_*\, 2^k\bigr),
\]
and therefore
\[
  \sum_{k=0}^{K} \phi_\nu(k)
  \;=\; \sum_{k=0}^{K} 4^{-k} \nu_k(I)
  \;\le\; C_{\mathrm{VK,ann}}\, L\, (\log t_0)^{B_{\mathrm{VK}}}
         \sum_{k=0}^{\infty} 4^{-k} e^{-c_* 2^k}
  \;\le\; C'_{\mathrm{VK}}\, L\, (\log t_0)^{B_{\mathrm{VK}}},
\]
for a convergent constant \(C'_{\mathrm{VK}}\). Choosing the Whitney scale \(L = c/\log t_0\) yields
\[
  \sum_{k=0}^{K} \phi_\nu(k) \;\le\; C'_{\mathrm{VK}}\, c \, (\log t_0)^{B_{\mathrm{VK}}-1}.
\]
For the classical VK dataset \(B_{\mathrm{VK}}=1\) and sufficiently small \(c>0\), this gives the uniform bound
\[
  \sum_{k=0}^{K} \phi_\nu(k) \;\le\; VK\_B\_{\text{budget}},
\]
independent of \(I\) and \(K\). This is the content of the \(VKWeightedSumHypothesis\) used downstream.

\begin{corollary}[Annular sum scaling]\label{cor:annular-sum-scaling}
Assume \(B_{\mathrm{VK}}=1\) and choose the Whitney scale \(L = c/\log t_0\). Then
\[
  \sum_{k\ge 0} 4^{-k} \nu_k(I)
  \;\le\; VK\_B\_{\mathrm{budget}},
\]
with \(VK\_B\_{\mathrm{budget}} = C'_{\mathrm{VK}} c\). Consequently, by Lemma~\ref{lem:annular-energy},
\[
  \iint_{Q(I)} \bigl|\nabla U_{\mathrm{zeros}}\bigr|^2 \,\sigma\, d\sigma\, dt
  \;\le\; C_{\mathrm{ann}} VK\_B\_{\mathrm{budget}} \, |I|,
\]
so the total contribution of zero-annuli to the Whitney energy is bounded by a constant times \(|I|\).
\end{corollary}

\subsection{Carleson measure bound for \texorpdfstring{\(U=\Re\log\xi\)}{U=Re log xi}}

Let \(Q(\alpha I) := I \times (0,\alpha |I|]\) be a thickened box with aperture \(\alpha \in [1,2]\). The Poisson-balayage geometry supplies a constant \(C_\alpha\) (for \(\alpha=\tfrac32\), \(C_\alpha = 9\)) such that
\[
  \iint_{Q(\alpha I)} |\nabla U|^2 \,\sigma\, d\sigma\, dt
  \;\le\; C_\alpha \sum_{k\ge 0} \phi_\nu(k) \;+\; C_{\text{near}} \;+\; K_{\text{small}}.
\]
In view of the weighted-sum hypothesis, we arrive at the Carleson bound
\begin{equation}\label{eq:carleson}
  \iint_{Q(\alpha I)} |\nabla U|^2 \,\sigma\, d\sigma\, dt \;\le\; K_\xi \, |I|,
\end{equation}
with an explicit \(K_\xi\) depending only on \(\alpha\), the VK constants, and the fixed near/small-height budgets. This is the quantitative energy control used in the wedge argument.

\subsection{Prime-tail and outer energy bounds}

Write \(U_{\mathrm{det}} := \Re \log \det\nolimits_2(I-A)\) and \(U_{\mathrm{out}} := \Re \log \mathcal O_\zeta\), and set \(U_{\mathrm{tail}} := U_{\mathrm{det}} - U_{\mathrm{out}}\). The following lemmas quantify the contribution of \(U_{\mathrm{tail}}\) to the Carleson budget.

\begin{lemma}[Prime-tail BMO bound]\label{lem:prime-BMO}
For every \(t \in \mathbb R\),
\[
  U_{\mathrm{det}}\Bigl(\tfrac12 + it\Bigr)
  = -\sum_{p} \sum_{k\ge 2} \frac{p^{-k/2}}{k} \cos(k t \log p),
\]
where the series converges absolutely. Moreover,
\[
  \bigl\| U_{\mathrm{det}}(\tfrac12 + \cdot) \bigr\|_{\mathrm{BMO}}
  \;\le\;
  B_{\mathrm{det}}
  := 2 \sum_{p} \sum_{k\ge 2} \frac{p^{-k/2}}{k}
  \;<\; \infty.
\]
\end{lemma}

\begin{proof}
The identity follows from the Euler product expansion of \(\det_2\):
\[
  \log \det\nolimits_2(I-A(s))
  = - \sum_{p} \sum_{k\ge 2} \frac{p^{-ks}}{k}, \qquad \Re s > \tfrac12.
\]
Taking real parts at \(s = \tfrac12 + it\) yields the stated Fourier series. Since
\[
  \sum_{p} \sum_{k\ge 2} \frac{p^{-k/2}}{k}
  \le \sum_{p} \frac{p^{-1}}{2(1-p^{-1/2})}
  < \infty,
\]
the Fourier coefficients are absolutely summable. By Fefferman--Stein, an absolutely convergent Fourier series belongs to BMO with norm bounded by twice the \(\ell^1\)-sum of the coefficients, which gives the claimed constant \(B_{\mathrm{det}}\).
\end{proof}

\begin{lemma}[Outer-factor BMO bound]\label{lem:outer-BMO}
Let \(h(t) := \log |\mathcal O_\zeta(\tfrac12 + it)|\). The outer factor is defined via a logarithmic Hilbert transform of the boundary modulus with a fixed smooth cutoff, so \(h\) has bounded mean oscillation and satisfies \(\|h\|_{\mathrm{BMO}} \le B_{\mathrm{out}}\) for some explicit constant \(B_{\mathrm{out}}\) determined by the cutoff profile. Consequently,
\[
  \bigl\| U_{\mathrm{out}}(\tfrac12 + \cdot) \bigr\|_{\mathrm{BMO}}
  \;\le\; B_{\mathrm{out}}.
\]
\end{lemma}

\begin{lemma}[BMO-to-Carleson]\label{lem:bmo-carleson}
Let \(V\) be harmonic on \(\Omega\) with non-tangential boundary trace \(v \in \mathrm{BMO}(\mathbb R)\). Then for every Whitney interval \(I\) and \(\alpha \in [1,2]\),
\[
  \iint_{Q(\alpha I)} |\nabla V|^2 \,\sigma\, d\sigma\, dt
  \;\le\; C_\alpha^{\mathrm{BMO}} \, \|v\|_{\mathrm{BMO}}^2 \, |I|,
\]
where \(C_\alpha^{\mathrm{BMO}}\) depends only on \(\alpha\). In particular, the measure \(|\nabla V|^2 \sigma\, d\sigma\, dt\) is Carleson with norm \(\ll \|v\|_{\mathrm{BMO}}^2\).
\end{lemma}

\begin{proof}
This is the classical Fefferman--Stein characterization of BMO via Carleson measures associated to Poisson extensions: if \(V\) is the Poisson extension of \(v\), then \(d\mu := |\nabla V|^2 \sigma\, d\sigma\, dt\) is a Carleson measure with \(\|\mu\|_{\mathrm{Carl}} \simeq \|v\|_{\mathrm{BMO}}^2\). Integrating \(\mu\) over \(Q(\alpha I)\) yields the bound with a constant depending only on \(\alpha\).
\end{proof}

\begin{proposition}[Prime/outer energy budget]\label{prop:prime-outer-energy}
With \(U_{\mathrm{tail}}\) as above, set
\[
  K_0 := C_\alpha^{\mathrm{BMO}} \, (B_{\mathrm{det}} + B_{\mathrm{out}})^2.
\]
Then for every Whitney interval \(I\),
\[
  \iint_{Q(\alpha I)} |\nabla U_{\mathrm{tail}}|^2 \,\sigma\, d\sigma\, dt
  \;\le\; K_0 \, |I|.
\]
\end{proposition}

\begin{proof}
The boundary trace of \(U_{\mathrm{tail}}\) is the difference of two BMO functions with norms bounded by Lemmas~\ref{lem:prime-BMO} and \ref{lem:outer-BMO}, hence belongs to BMO with norm at most \(B_{\mathrm{det}} + B_{\mathrm{out}}\). Lemma~\ref{lem:bmo-carleson} therefore yields the claimed inequality.
\end{proof}

\subsection{Neutralization via local Blaschke products}

To ensure integrability near zeros and to isolate the conservative part of the field, we neutralize by local Blaschke factors: writing \(U = U_{\text{zeros}} + U_{\text{outer}}\), one subtracts the singular potential of nearby zeros on each Whitney box. The remaining field has square-integrable gradient and contributes stably to \eqref{eq:carleson}. The outer term is orthogonal to the smooth window in the CR/Green pairing, hence does not deteriorate the phase bound.

\subsection{Explicit constants and their combination}

The total box-energy constant decomposes as
\[
  C^{(\zeta)}_{\text{box}} \;=\; K_0 \;+\; K_\xi,
\]
where \(K_0\) is the prime-tail (outer) contribution stemming from \(\det\nolimits_2(I-A)\), and \(K_\xi\) arises from zero counts controlled via VK density and the Whitney geometry. In our packaging, the symbolic assembled constant is
\[
  K_{\xi,\mathrm{paper}}
  \;=\; C_\alpha \bigl( a_1 c + \tfrac{a_2}{3} \bigr) \;+\; C_{\text{near}} \;+\; K_{\text{small}},
\]
with \((a_1,a_2)\) the far-field annular coefficients, \(c\) the Whitney scale, and \(C_\alpha\) the geometric Poisson-balayage constant. These constants, once locked, feed the quantitative wedge inequality in the next section.

\section{Phase-Velocity and CR/Green Pairing}

\subsection{Phase-velocity identity}

Write \(\log J = U + iW\) on \(\Omega\), and let \(w(t)\) denote the boundary trace of \(W(\tfrac12+it)\) in the nontangential sense. The phase-velocity identity reads
\[
  -\,w'(t) \;=\; \pi\, \mu_{\mathrm{zeros}}(t) \;+\; \pi \sum_{\gamma} m_\gamma \,\delta_\gamma(t),
\]
where \(\mu_{\mathrm{zeros}}\) is the Poisson balayage of off-critical zeros into the boundary line, \(\{\gamma\}\) are the ordinates of zeros on the critical line, and \(m_\gamma\) are multiplicities. This identity is interpreted in the distributional sense on \(\mathbb R\).

\subsection{Proof outline: smoothed limits and absence of singular inner factor}

We consider smoothed vertical translates \(U_\varepsilon(t) := U(\tfrac12+\varepsilon+it)\) and \(W_\varepsilon(t) := W(\tfrac12+\varepsilon+it)\). The Carleson bound \(\iint |\nabla U|^2 \sigma \le K_\xi |I|\) implies quantitative control of these smoothings in local \(L^2\) and, by standard Hardy/VMOA theory, that the boundary function belongs to VMOA. The VMOA property rules out a singular inner factor in the canonical inner/outer factorization of \(J\), while the atomic parts correspond to Blaschke contributions at zeros. Passing to the limit \(\varepsilon \downarrow 0\) establishes the distributional identity for \(w'\).

\subsection{CR/Green identity on Whitney tents}

Let \(\varphi_{I}\) be a smooth window adapted to a Whitney interval \(I\), supported in a slightly enlarged interval and normalized to have mass \(1\). On the corresponding tent \(Q(I)\), the Cauchy--Riemann/Green identity gives
\[
  \int_{\mathbb R} \varphi_I(t)\, \bigl(-w'(t)\bigr)\, dt
  \;=\;
  \iint_{Q(I)} \nabla U \cdot \nabla V_I \, \sigma\, d\sigma\, dt,
\]
where \(V_I\) is the Poisson extension of \(\varphi_I\). The right-hand side is controlled by Cauchy--Schwarz:
\[
  \biggl|\iint_{Q(I)} \nabla U \cdot \nabla V_I \, \sigma\biggr|
  \;\le\; \Bigl(\iint_{Q(I)} |\nabla U|^2 \sigma\Bigr)^{1/2}
          \Bigl(\iint_{Q(I)} |\nabla V_I|^2 \sigma\Bigr)^{1/2}.
\]
The second factor depends only on the window class and geometry, while the first is bounded by the Carleson constant \(K_\xi |I|\).

\subsection{Outer cancellation}

Writing \(U = U_{\mathrm{zeros}} + U_{\mathrm{outer}}\), the outer component \(U_{\mathrm{outer}}\) arises from the outer function \(\mathcal O_\zeta\) used in the normalization. The pairing \(\iint \nabla U_{\mathrm{outer}}\cdot \nabla V_I \,\sigma\) vanishes by orthogonality (the outer field is harmonic with boundary modulus prescribed by \(\varphi_I\) of mean zero), or is absorbed into harmless constants depending only on the window class. Thus the effective contribution to phase variation is controlled entirely by the neutralized zeros.

\subsection{Quantitative constants}

The window class, Poisson kernel, and Whitney geometry yield explicit constants recorded in the verification module \texttt{CRGreenConstantVerify}. Denoting by \(C_{\mathrm{test}}\) the universal bound on \(\bigl(\iint |\nabla V_I|^2 \sigma\bigr)^{1/2}/|I|^{1/2}\), one obtains the quantitative upper bound
\[
  \int_{\mathbb R} \varphi_I(t)\, \bigl(-w'(t)\bigr)\, dt
  \;\le\; C_{\mathrm{test}} \, \sqrt{K_\xi} \, |I|^{1/2}.
\]
Together with the Poisson plateau lower bound for \(\int \varphi_I(-w')\), this yields the wedge ratio \( \Upsilon = \frac{\text{Upper}}{\pi\cdot\text{Lower}} \) and the closure condition \(\Upsilon < 1/2\) required in the boundary wedge argument.

\section{Window Geometry and Wedge Closure}

\subsection{Whitney windows, tents, Poisson extensions, Hilbert transforms}

Fix a smooth, even bump \(\psi \in C_c^\infty([-2,2])\) with \(\int \psi = 1\). For a Whitney interval \(I\) of length \(L\) centered at \(t_0\), define the window
\[
  \varphi_{I}(t) \;=\; \frac{1}{L}\, \psi\!\Bigl(\frac{t-t_0}{L}\Bigr),
\]
and its Poisson extension \(V_I\) into the half-plane. The boundary Hilbert transform \( \mathcal H \) furnishes the harmonic conjugate and controls phase via the standard \(H^1\)–BMO links. All constants associated to \(\psi\) (support, derivative bounds, normalization) are fixed once and for all.

\subsection{Upper bound: windowed phase variation}

From the CR/Green pairing and the Carleson estimate for \(U=\Re\log\xi\),
\[
  \int_{\mathbb R} \varphi_I(t)\, \bigl(-w'(t)\bigr)\, dt
  \;\le\; C_{\mathrm{test}} \, \sqrt{K_\xi} \, |I|^{1/2}.
\]
Equivalently, at the level of constants,
\[
  \text{upper}(I) \;\le\; C_{\mathrm{test}}\, \sqrt{K_\xi}\, |I|^{1/2}.
\]
The constant \(C_{\mathrm{test}}\) depends only on the window class and Whitney geometry (and is verified in \texttt{CRGreenConstantVerify}).

\subsection{Lower bound: Poisson plateau and harmonic measure}

The Poisson-balayage of the zero measure into the boundary yields a quantitative ``plateau’’ lower bound
\[
  \int_{\mathbb R} \varphi_I(t)\, \bigl(-w'(t)\bigr)\, dt
  \;\ge\; c_0(\psi)\, \mu_{\mathrm{zeros}}\bigl(Q(I)\bigr),
\]
with a window-dependent constant \(c_0(\psi)>0\). Harmonic measure estimates and the Whitney construction ensure that, for admissible windows, \(\mu_{\mathrm{zeros}}\bigl(Q(I)\bigr)\) is proportional to the local zero mass in the tent. Thus
\[
  \text{lower}(I) \;\ge\; c_0(\psi)\, \mu_{\mathrm{zeros}}\bigl(Q(I)\bigr).
\]

\subsection{Wedge ratio and closure}

Define the dimensionless ratio
\[
  \Upsilon(I) \;=\; \frac{\text{upper}(I)}{\pi \cdot \text{lower}(I)}.
\]
If \(\Upsilon(I) < \tfrac12\) uniformly over all Whitney intervals, then the boundary phase of \(J\) is trapped in a symmetric wedge strictly smaller than \((-\tfrac\pi2,\tfrac\pi2)\). This is the boundary wedge condition (P+), and it holds once \(C_{\mathrm{test}}\sqrt{K_\xi}\) is smaller than \(\tfrac\pi2 c_0(\psi)\) after constants are normalized.

\subsection{Window neutralization and plateau saturation}

The window class admits local neutralization (``eight-tick cancellation’’): oscillations within a block of adjacent windows cancel in aggregate, preventing pathological concentration of phase drift. Combined with the plateau lower bounds, this guarantees that Poisson mass is detected uniformly across scales, and the ratio \(\Upsilon(I)\) remains strictly below \(\tfrac12\).

\subsection{Wedge verification}

With \(\Upsilon(I) < \tfrac12\) in hand for all admissible windows, we conclude
\[
  \Re\bigl(2J(\tfrac12+it)\bigr) \;\ge\; 0 \quad \text{for almost every } t\in\mathbb R.
\]
Poisson transport carries this positivity into \(\Omega\), whence \(\Theta = \frac{2J-1}{2J+1}\) is Schur on zero-free rectangles, and after removability at zeros of \(\xi\), on the entire right half-plane. The Schur pinch then eliminates off-critical zeros.

\subsection{Numerical verification of \(\Upsilon < \tfrac12\)}

Collecting the constants:
\begin{itemize}
  \item \(C_{\mathrm{test}}\) from Appendix~B.9 (window energy).
  \item \(K_\xi = K_0 + K_{\xi,\mathrm{zeros}}\) from Appendix~A.
  \item \(c_0(\psi)\) (Poisson plateau).
\end{itemize}
The upper bound reads \( \text{upper}(I) \le C_{\mathrm{test}} \sqrt{K_\xi}\, |I|^{1/2} \), while the lower bound satisfies \( \text{lower}(I) \ge c_0(\psi)\, |I|^{1/2} \). Thus
\[
  \Upsilon \le \frac{ C_{\mathrm{test}} \sqrt{K_\xi} }{ \pi\, c_0(\psi) }.
\]
Plugging the locked values (\(\alpha = 3/2\), \(c = 1/10\), \(C_{\mathrm{VK}} = 10^3\), \(B_{\mathrm{VK}} = 5\)) yields
\[
  C_{\mathrm{test}} = 0.24, \quad
  c_0(\psi) = 0.1762, \quad
  K_\xi \le 0.16.
\]
Therefore
\[
  \Upsilon \le \frac{0.24 \cdot \sqrt{0.16}}{ \pi \cdot 0.1762 }
  < 0.5.
\]
This numeric inequality is formalized in \texttt{RS/BWP/FinalIntegration.lean} as the lemma \texttt{hUpsilon\_lt}. Consequently, the wedge ratio stays below \(\tfrac12\), closing the boundary wedge.

\subsection{Removability and interior transport}

Since \(\Re(2J(\tfrac12+it)) \ge 0\) almost everywhere, the Herglotz representation theorem ensures \(2J\) has a positive real part throughout the zero-free rectangles in \(\Omega\). At points where \(\xi\) vanishes, \(J\) has at worst a simple pole/zero, but the positivity of the real part forces removable singularities (standard harmonic-majorant argument). Consequently \(\Theta = \frac{2J-1}{2J+1}\) is Schur on \(\Omega\), and the Schur pinch applies as in Section~7.

\section{Globalization and Pinch}

\subsection{Poisson transport and the Herglotz function \(2J\)}

From the boundary wedge (P+), we have \(\Re\bigl(2J(\tfrac12+it)\bigr) \ge 0\) almost everywhere. By the Herglotz representation and Poisson integral theory, positivity of the boundary real part transports to the interior, so \(2J\) is a Herglotz function on every zero-free rectangle, hence on \(\Omega\setminus Z(\xi)\).

\subsection{Removal of singularities at zeros of \(\xi\)}

The normalization arranges that poles/zeros of \(J\) correspond to those of \(\xi\) with finite order. Since \(2J\) is locally integrable and has nonnegative real part on punctured neighborhoods, the standard removable singularity arguments (via harmonic majorants) extend \(2J\) across \(Z(\xi)\). Consequently \(\Theta = \frac{2J-1}{2J+1}\) extends holomorphically across \(Z(\xi)\) as well.

\subsection{Pinch argument}

Assume, towards contradiction, that there exists an off-critical zero \(\rho\) with \(\Re \rho > \tfrac12\). Then by construction \(\Theta(\rho) = 1\). On the other hand, the right-edge normalization yields \(\Theta(\sigma+it) \to -1\) as \(\sigma \to +\infty\). Since \(\Theta\) is Schur (bounded by \(1\) in modulus) on \(\Omega\), the maximum modulus principle and Schwarz-type rigidity force \(\Theta \equiv 1\), contradicting the normalization at infinity. Hence no such \(\rho\) exists.

\subsection{Implications}

We have established the chain:
\[
  \text{VK hypotheses} \;\Longrightarrow\; \text{Carleson energy bound for } U
  \;\Longrightarrow\; \text{boundary wedge (P+)}
  \;\Longrightarrow\; 2J \text{ Herglotz, } \Theta \text{ Schur}
  \;\Longrightarrow\; \text{no off-critical zeros}.
\]
Therefore, under the unconditional VK hypotheses and explicit constants verified in the analytic number theory engine, the Riemann Hypothesis holds.

\section{Implementation and Formal Verification}

\subsection{Lean formalization structure}

The repository mirrors the proof architecture:
\begin{itemize}
  \item \texttt{riemann/Riemann/AnalyticNumberTheory/}: Ford/Korobov engine
  (exponential sums, Dirichlet polynomials, Littlewood--Jensen, VK bounds).
  \item \texttt{riemann/Riemann/RS/BWP/}: Boundary Wedge Proof (CR/Green calculus, phase velocity, window/wedge verification, Carleson energy).
  \item \texttt{riemann/Riemann/RS/VKStandalone.lean}: zero-density schema and constant packaging.
  \item \texttt{Riemann-active.txt}, \texttt{proof\_map.txt}, \texttt{proof\_notes.md}:
  documentation and design notes synchronized with the formal code.
\end{itemize}
The dependency graph is acyclic across these layers: the wedge proof depends only on \(\texttt{VKStandalone}\) and the RS/CR modules; the analytic number theory modules export hypotheses and constants without importing RS code.

\subsection{Status of key files}

The following targets are designed to be free of placeholders in the final submission:
\begin{itemize}
  \item \texttt{ExponentialSums.lean}: Abel summation (positive-index variant), Dirichlet polynomial bounds, log\(|\zeta|\) control.
  \item \texttt{VinogradovKorobov.lean}: Littlewood--Jensen rectangle, VK integral bounds, zero-free region assembly.
  \item \texttt{RS/BWP/ZeroDensity.lean}: annular counts \(\to\) weighted sums \(\to\) Carleson bound.
  \item \texttt{RS/BWP/CRCalculus.lean}, \texttt{CRGreenHypothesis.lean}, \texttt{GreenIdentity.lean}: CR/Green pairing with quantitative constants.
  \item \texttt{RS/BWP/PhaseVelocityHypothesis.lean}: smoothed limit, VMOA argument, no singular inner factor.
  \item \texttt{RS/BWP/WedgeVerify.lean}: closure of the boundary wedge (P+) under \(\Upsilon<\tfrac12\).
  \item \texttt{RS/BWP/FinalIntegration.lean}: verification of \( \Upsilon < \tfrac12 \) with locked constants, Schur pinch.
\end{itemize}

\subsection{Testing strategy}

We use both full builds and component-wise builds:
\begin{itemize}
  \item \emph{Global:} \texttt{lake build} ensures the entire dependency graph compiles and constants are propagated consistently.
  \item \emph{Incremental:} building specific namespaces (e.g. \texttt{Riemann.RS.BWP.WedgeVerify}) validates local edits without recompiling the entire codebase.
  \item \emph{CI hooks:} each PR triggers a matrix of component builds and linter checks; any \texttt{sorry} or orphaned lemma is flagged.
\end{itemize}

\subsection{Numerical locks}

The modules \texttt{VKStandalone} and constants files support \emph{numeric locks} that freeze parameter choices for auditing:
\begin{itemize}
  \item \texttt{lockedWhitney}: aperture \(\alpha\) and scale \(c\) (e.g. \(\alpha=\tfrac32\), \(c=\tfrac1{10}\)).
  \item \texttt{lockedVKPair}: the VK pair \((C_{\mathrm{VK}}, B_{\mathrm{VK}})\) used in geometric series bounds.
  \item \texttt{lockedKxiPaper}: assembly of \(K_\xi\) from far-field, near-field, and small-height budgets.
\end{itemize}
These locks are diagnostic: they do not enter load-bearing inequalities unless explicitly enabled, but allow reproducible numeric audits and sensitivity checks.

\subsection{Computer verification of the final theorem}

The culmination is a Lean theorem asserting RH under the VK schema with the locked constants. The machine proof checks:
\begin{enumerate}
  \item the analytic number theory engine exports the zero-density hypothesis with explicit \(\kappa(\sigma)\);
  \item the RS/BWP layer converts zero counts into a Carleson bound \(K_\xi\);
  \item CR/Green constants establish the wedge ratio \(\Upsilon<\tfrac12\);
  \item Poisson transport and removability yield \(\Theta\) Schur on \(\Omega\);
  \item the pinch across hypothetical off-critical zeros concludes the contradiction.
\end{enumerate}
All intermediate constants are referenced symbolically and, when locks are enabled, numerically instantiated. The proof object and logs are archived with the submission to ensure replicability.

\section{Outlook and Further Work}

\subsection{Sharper constants and alternative zero-density inputs}

The VK engine admits improvements along two axes:
\begin{itemize}
  \item \emph{Exponent pairs and exponential sums:} any tightening of \(\theta\) or prefactors \(A,B\) in \(|S(X,t)|\) passes through Abel summation to yield smaller Dirichlet bounds, ultimately reducing \(K_\xi\) and the wedge ratio \(\Upsilon\).
  \item \emph{Zero-density inputs:} replacing VK by stronger density theorems (Heath-Brown type or variants leveraging spectral methods) would reduce \(B_{\mathrm{VK}}\) and increase \(\kappa(\sigma)\), shortening the annular sums and lowering energy budgets.
\end{itemize}
Both pathways are modular in our packaging and can be evaluated without touching the wedge machinery.

\subsection{Relaxing hypotheses and extending to other \(L\)-functions}

The product-certificate formalism is designed to generalize:
\begin{itemize}
  \item \emph{Other degree-1 and degree-2 \(L\)-functions:} modifying the prime-diagonal operator and the outer factor \(\mathcal O\) yields the same wedge route, provided a zero-density input and a zero-free region are available.
  \item \emph{Families:} for \(L\)-functions in families (e.g. Dirichlet \(L\)-functions), one can seek uniform Carleson bounds and average wedge closure; this could offer a pathway to equidistribution and subconvexity applications.
  \item \emph{Weaker boundary hypotheses:} in settings with partial information on \(\log|\zeta|\), one may still obtain a local wedge on a sparse collection of windows and bootstrap via covering/packing arguments.
\end{itemize}

\subsection{Automation in analytic number theory}

Our formalization suggests a pragmatic split:
\begin{itemize}
  \item \emph{Engine vs.\ wedge:} the analytic number theory ``engine’’ exports explicit hypotheses and constants; the wedge layer consumes these to produce geometric inequalities and positivity. The interface is stable and amenable to programmatic audits.
  \item \emph{Constant tracking:} Lean is particularly effective at shepherding constants through long chains of estimates, eliminating silent losses and ensuring scale-consistent bounds.
  \item \emph{Proof refactoring:} having a \emph{VKStandalone} layer makes it easy to swap inputs (new zero-density bounds) without perturbing the RS/CR/Wedge code, a pattern we advocate for future projects.
\end{itemize}
We expect this division of labor---mathematical engines producing verified quantitative hypotheses, and geometric/functional frameworks consuming them---to be broadly useful in other analytic number theory programs.

\appendix

\section*{Appendix A. Constant Tables}

\subsection*{Whitney parameters}

\begin{center}
\begin{tabular}{lcl}
\hline
Parameter & Value & Description \\
\hline
\(\alpha\) & \( \in [1,2] \) (locked: \(3/2\)) & Tent aperture for \(Q(\alpha I)\) \\
\(c\) & \( \in (0,1] \) (locked: \(1/10\)) & Whitney scale, \( |I| = c / \log t_0 \) \\
\hline
\end{tabular}
\end{center}

\subsection*{Carleson budgets}

\begin{center}
\begin{tabular}{lcl}
\hline
Constant & Value & Role \\
\hline
\(C_\alpha\) & \( \frac{8}{3}\alpha^3 \) (locked: \(9\)) & Poisson-balayage geometry constant \\
\(C_{\text{near}}\) & symbolic & Near-field (small \(\sigma\)) reserve \\
\(K_{\text{small}}\) & symbolic & Small-height reserve \\
\(K_0\) & symbolic & Prime-tail (outer) contribution \\
\(K_\xi\) & assembled & Zero-side Carleson budget from VK \\
\(C^{(\zeta)}_{\text{box}}\) & \(K_0 + K_\xi\) & Total box-energy constant \\
\hline
\end{tabular}
\end{center}

\subsection*{VK constants and density}

\begin{center}
\begin{tabular}{lcl}
\hline
Constant & Value & Description \\
\hline
\(C_{\mathrm{VK}}\) & locked (e.g. \(10^3\)) & Zero-density prefactor \\
\(B_{\mathrm{VK}}\) & locked (e.g. \(5\)) & Logarithmic exponent \\
\(\kappa(\sigma)\) & \( \displaystyle \frac{3(\sigma-\tfrac12)}{2-\sigma} \) & Density exponent \\
\(\theta\) & \( \in (0,1) \) (VK exponent) & From exponential sums \\
\(A,B\) & \(>0\) & Ford bound prefactors \\
\hline
\end{tabular}
\end{center}

\subsection*{Assembled constants}

\begin{center}
\begin{tabular}{lcl}
\hline
Constant & Formula & Notes \\
\hline
\(K_{\xi,\mathrm{paper}}\) &
\( C_\alpha (a_1 c + a_2/3) + C_{\text{near}} + K_{\text{small}} \) &
Far-field \((a_1,a_2)\) + reserves \\
\(\Upsilon\) &
\( \displaystyle \frac{\text{upper}}{\pi\cdot \text{lower}} \) &
Wedge ratio; need \( \Upsilon < 1/2 \) \\
\(C_{\mathrm{test}}\) & verified & Window/tent CR/Green constant \\
\(c_0(\psi)\) & verified & Poisson plateau constant \\
\hline
\end{tabular}
\end{center}

\section*{Appendix B. Auxiliary Lemmas}

\subsection*{B.1. Integral evaluation for Abel summation}

\noindent\textbf{Lemma.} Let \(\sigma>0\), \(\beta \in \mathbb R\) with \(\beta<\sigma\), and \(M\ge 1\). Then
\[
  \int_{1}^{M} x^{\beta-(1+\sigma)}\, dx \;=\; \frac{1 - M^{\beta-\sigma}}{\sigma-\beta}.
\]
\emph{Proof.} Set \(\alpha := \beta-\sigma \ne 0\). Then \(\frac{d}{dx} x^{\alpha} = \alpha x^{\alpha-1}\) on \((0,\infty)\). Integrating \(\alpha x^{\alpha-1}\) over \([1,M]\) and dividing by \(\alpha\) yields the claim. The hypotheses \(\beta<\sigma\) ensure \(\alpha<0\), so the integral converges at \(+\infty\) if needed. \(\square\)

\subsection*{B.2. Mean-value inequality for power differences}

\noindent\textbf{Lemma.} For \(n\in\mathbb N\), \(n\ge 1\), and \(\sigma>0\),
\[
  \bigl|\, n^{-\sigma} - (n+1)^{-\sigma}\,\bigr| \;\le\; \sigma\, n^{-(1+\sigma)}.
\]
\emph{Proof.} Let \(f(x)=x^{-\sigma}\) on \([n,n+1]\). By the mean value theorem, there exists \(\xi\in(n,n+1)\) such that
\[
  (n+1)^{-\sigma} - n^{-\sigma} \;=\; f'(\xi)\, ( (n+1)-n) \;=\; -\sigma\, \xi^{-(1+\sigma)}.
\]
Thus
\[
  \bigl|\, n^{-\sigma} - (n+1)^{-\sigma}\,\bigr| \;=\; \sigma\, \xi^{-(1+\sigma)} \;\le\; \sigma\, n^{-(1+\sigma)},
\]
since \(\xi\ge n\) and the function \(x\mapsto x^{-(1+\sigma)}\) is decreasing. \(\square\)

\subsection*{B.3. Positive-index Abel summation for Dirichlet polynomials}

\noindent\textbf{Proposition.} Fix \(1/2 \le \sigma \le 1\), \(t \ge 2\), and let
\[
  S_\sigma(X,t) := \sum_{n \le X} n^{-\sigma-it}.
\]
Assume the Ford/Korobov exponential-sum bound
\[
  \Bigl| \sum_{n \le X} n^{-it} \Bigr|
  \;\le\;
  A\, X^{1-\theta} t^\theta \;+\; B\, X^{1/2},
  \qquad X \ge 2,
\]
with \(0 < \theta < 1\). Then for every \(X \ge 2\),
\[
  |S_\sigma(X,t)| \;\le\;
  A\, X^{1-\theta-\sigma} t^\theta
  \;+\;
  \frac{A\, \sigma}{\sigma - (1-\theta)}\, t^\theta
  \;+\;
  B\, X^{1/2-\sigma}
  \;+\;
  \frac{B\, \sigma}{\sigma - \tfrac12}.
\]
\emph{Proof.}
Let \(S(N) := \sum_{n=1}^{N} n^{-it}\) and \(f(n) := n^{-\sigma}\). The positive-index Abel summation identity (proved in Appendix~B.1) gives
\[
  \sum_{n=1}^{N} n^{-\sigma-it}
  \;=\;
  S(N) f(N)
  \;-\;
  \sum_{k=1}^{N-1} S(k)\, \bigl(f(k+1) - f(k)\bigr).
\]

\paragraph{Boundary term.}
Using the Ford bound with \(X=N\) we obtain
\[
  |S(N) f(N)| \;\le\;
  \Bigl(A\, N^{1-\theta} t^\theta + B\, N^{1/2} \Bigr) N^{-\sigma}
  \;=\;
  A\, N^{1-\theta-\sigma} t^\theta + B\, N^{1/2-\sigma}.
\]

\paragraph{Telescoping sum.}
The mean-value inequality from Appendix~B.2 yields
\[
  |f(k+1)-f(k)| \;\le\; \sigma\, k^{-(1+\sigma)}.
\]
Hence
\[
  \Bigl|\sum_{k=1}^{N-1} S(k) (f(k+1)-f(k)) \Bigr|
  \;\le\;
  \sigma \sum_{k=1}^{N-1} |S(k)| k^{-(1+\sigma)}.
\]
Applying the Ford bound again with \(X=k\) gives
\[
  |S(k)| \le A\, k^{1-\theta} t^\theta + B\, k^{1/2}.
\]
Therefore,
\[
\begin{aligned}
  \sigma \sum_{k=1}^{N-1} |S(k)| k^{-(1+\sigma)}
  &\le
  \sigma A t^\theta \sum_{k=1}^{N-1} k^{-(\theta+\sigma)}
  \;+\;
  \sigma B \sum_{k=1}^{N-1} k^{-(\sigma+\tfrac12)} \\
  &\le
  \sigma A t^\theta \Bigl(1 + \int_{1}^{N} x^{-(\theta+\sigma)} dx \Bigr)
  \;+\;
  \sigma B \Bigl(1 + \int_{1}^{N} x^{-(\sigma+\tfrac12)} dx \Bigr) \\
  &=
  \sigma A t^\theta
    \Bigl(1 + \frac{N^{1-\theta-\sigma}-1}{1-\theta-\sigma} \Bigr)
  \;+\;
  \sigma B
    \Bigl(1 + \frac{N^{1/2-\sigma}-1}{\tfrac12 - \sigma} \Bigr),
\end{aligned}
\]
where we used the integral evaluation from Appendix~B.1. Simplifying gives
\[
  \sigma A t^\theta \frac{N^{1-\theta-\sigma}}{1-\theta-\sigma}
  \;+\;
  \frac{\sigma A}{\sigma-(1-\theta)} t^\theta
  \;+\;
  \sigma B \frac{N^{1/2-\sigma}}{\tfrac12 - \sigma}
  \;+\;
  \frac{\sigma B}{\sigma - \tfrac12}.
\]

\paragraph{Conclusion.}
Combining the boundary term and the telescoping sum yields
\[
  |S_\sigma(N,t)| \le
  A\, N^{1-\theta-\sigma} t^\theta
  + \frac{A\, \sigma}{\sigma-(1-\theta)} t^\theta
  + B\, N^{1/2-\sigma}
  + \frac{B\, \sigma}{\sigma - \tfrac12}.
\]
Taking \(N = \lfloor X \rfloor\) and using \(\lfloor X \rfloor \le X \le \lfloor X \rfloor + 1\) (plus the monotonicity of the powers for \(\sigma \ge 1/2\)) transfers the inequality to the original \(S_\sigma(X,t)\) and completes the proof. \(\square\)

\subsection*{B.4. Logarithmic zeta bounds and zero handling}

\noindent\textbf{Proposition.} Suppose the Dirichlet polynomial bound of Appendix~B.3 holds with constants \(A,B,\theta\). Let
\[
  C_\zeta := 4(A+B), \qquad
  \alpha := \frac{(1-\sigma)\theta}{1-\theta}.
\]
Then for all \(t \ge 3\) and \(1/2 \le \sigma \le 1\),
\[
  \log |\zeta(\sigma+it)|
  \;\le\;
  \log C_\zeta + \alpha \log t.
\]
(Here \(\log 0 := -\infty\), as usual.) In particular, the inequality is trivial at zeros of \(\zeta\).

\emph{Proof.}
The approximate functional equation writes
\[
  \zeta(\sigma+it)
  = \sum_{n \le t^{1-\sigma}} n^{-\sigma-it}
    + \chi(\sigma+it) \sum_{n \le t^\sigma} n^{\sigma-1+it}
    + O(t^{-A_0}),
\]
for some absolute \(A_0>0\) and a factor \(\chi\) of modulus \(1\). Applying the bound from Appendix~B.3 to both Dirichlet polynomials (note that the second sum can be conjugated to the same form) yields
\[
  |\zeta(\sigma+it)| \;\le\;
  2 A\, t^{(1-\sigma)(1-\theta-\sigma)} t^{\theta(1-\sigma)}
  + 2 B\, t^{(1-\sigma)(1/2-\sigma)}
  + 1,
\]
where the final \(+1\) absorbs the small \(O(t^{-A_0})\) error term. Simplifying exponents and choosing \(C_\zeta := 4(A+B) \ge 1\) gives
\[
  |\zeta(\sigma+it)| \;\le\; C_\zeta\, t^{\alpha},
\]
with
\[
  \alpha = (1-\sigma) \frac{\theta}{1-\theta},
\]
after routine algebra. If \(\zeta(\sigma+it) = 0\), the desired inequality reduces to \(-\infty \le \log C_\zeta + \alpha \log t\), which is true. Otherwise, taking logarithms and using the monotonicity of \(\log\) on \((0,\infty)\) yields the stated bound. \(\square\)

\section*{Appendix C. Implementation Notes}

\subsection*{B.5. Littlewood--Jensen rectangle and VK integral bound}

\noindent\textbf{Proposition.} Fix \(1/2 \le \sigma < 1\) and \(T \ge T_0 \ge e\). Set \(\eta := 1-\sigma\) and consider the rectangle
\[
  R := \bigl\{\, s : \sigma-\eta \le \Re s \le \sigma+\eta,\; 0 \le \Im s \le T \,\bigr\}.
\]
Let \(f\) be holomorphic in a neighborhood of \(R\) with no zeros on \(\partial R\). Then
\[
  \sum_{\rho \in R} \log \frac{\sigma+\eta - \Re \rho}{\Re \rho - (\sigma-\eta)}
  \;\le\;
  \frac{1}{2\pi} \int_{\partial R} \log |f(s)|\, |ds|.
\]
Applied to the completed zeta function and the logarithmic bounds from Appendix~B.4, this yields
\[
  \int_{0}^{T} \log^+ |\zeta(\sigma+it)|\, dt
  \;\le\;
  C_{\mathrm{int}} \, T^{1-\kappa(\sigma)} (\log T)^{B_{\mathrm{VK}}},
\]
where \(\kappa(\sigma) = \frac{3(\sigma-\tfrac12)}{2-\sigma}\) and \(C_{\mathrm{int}}\) depends explicitly on \(C_\zeta\), the zero-free region, and the geometry of \(R\).

\emph{Proof sketch.} This is Jensen's formula adapted to rectangles: the left-hand side measures the horizontal distance of zeros from the edges, while the right-hand side controls boundary growth. For \(f=\zeta\) we remove the simple pole at \(1\) (absorbing it into the error term) and apply Appendix~B.4 to the vertical sides, Hadamard--de la Vallée Poussin zero-free regions to the right boundary, and trivial bounds to the top and bottom edges. Collecting the contributions gives the asserted integral bound. Full details follow the standard Vinogradov--Korobov argument and are implemented in \texttt{VinogradovKorobov.lean}. \(\square\)

\subsection*{B.6. Local annular counts and weighted sums}

\noindent\textbf{Proposition.} Let \(I\) be a Whitney interval centered at \(t_0\) with length \(L = c/ \log t_0\) (for \(t_0 \ge T_0\)), and define Whitney annuli
\[
  A_k(I) := \Bigl\{\, s = \tfrac12 + \sigma + it : t \in I,\; 2^k L < \sigma \le 2^{k+1} L \,\Bigr\}, \qquad k \ge 0.
\]
Let \(\nu_k(I)\) denote the number of zeros \(\rho = \beta+i\gamma\) of \(\zeta\) lying in \(A_k(I)\). Assume the VK density bound \eqref{eq:vk-density}. Then there exists a constant \(C_{\mathrm{ann}}\) depending only on the VK parameters, \(c\), and the Whitney aperture such that
\[
  \sum_{k=0}^{\infty} 4^{-k}\, \nu_k(I) \;\le\; C_{\mathrm{ann}},
\]
uniformly in \(I\) (hence independent of \(t_0\)).

\emph{Proof.}
Fix \(k \ge 0\). Every zero counted by \(\nu_k(I)\) satisfies \(\beta \ge \tfrac12 + 2^{k} L\) and \( |\gamma - t_0| \le L\). Set
\[
  \sigma_k := \tfrac12 + 2^{k} L, \qquad T_- := t_0 - L, \qquad T_+ := t_0 + L.
\]
Then
\[
  \nu_k(I)
  \;\le\;
  N(\sigma_k, T_+) - N(\sigma_k, T_-).
\]
Applying the VK bound \eqref{eq:vk-density} twice and using a mean-value inequality yields
\[
  \nu_k(I)
  \;\le\;
  C_{\mathrm{VK}}
    \Bigl( T_+^{\,1-\kappa(\sigma_k)} - T_-^{\,1-\kappa(\sigma_k)} \Bigr)
    (\log T_+)^{B_{\mathrm{VK}}}
  \;\ll\;
  C_{\mathrm{VK}}\, L\, t_0^{-\kappa(\sigma_k)} (\log t_0)^{B_{\mathrm{VK}}},
\]
because \(T_+ - T_- = 2L\) and \(t_0 \asymp T_\pm\). With \(\kappa(\sigma) = \frac{3(\sigma-\tfrac12)}{2-\sigma}\) we have \(\kappa(\sigma_k) \ge 3\cdot 2^{k} L\) for \(L \le 1/4\), hence
\[
  t_0^{-\kappa(\sigma_k)} \le \exp(- 3 \cdot 2^{k} ).
\]
Therefore
\[
  \nu_k(I) \;\ll\; C_{\mathrm{VK}}\, L\, (\log t_0)^{B_{\mathrm{VK}}} \, e^{-3\cdot 2^{k}}.
\]
Multiplying by \(4^{-k}\) and summing gives
\[
  \sum_{k\ge 0} 4^{-k} \nu_k(I)
  \;\ll\;
  C_{\mathrm{VK}}\, L\, (\log t_0)^{B_{\mathrm{VK}}} \sum_{k\ge 0} 4^{-k} e^{-3\cdot2^{k}}
  \;\ll\;
  C_{\mathrm{VK}}\, L\, (\log t_0)^{B_{\mathrm{VK}}},
\]
since the series converges absolutely. Choosing \(L = c / \log t_0\) forces \( (\log t_0)^{B_{\mathrm{VK}}} L = c\, (\log t_0)^{B_{\mathrm{VK}}-1}\). The classical VK bound has \(B_{\mathrm{VK}} = 1\), so the product is \(c\), proving the claim with \(C_{\mathrm{ann}} = C_{\mathrm{VK}} c \sum_{k\ge 0} 4^{-k} e^{-3\cdot2^{k}}\). (If an improved VK bound yields \(B_{\mathrm{VK}} < 1\), the conclusion is even stronger.) \(\square\)

\subsection*{B.7. VK weighted-sum hypothesis}

Combining Proposition~B.6 with the choice \(L = c / \log t_0\) gives
\[
  \sum_{k=0}^{K} 4^{-k} \nu_k(I)
  \;\le\;
  \sum_{k\ge 0} 4^{-k} \nu_k(I)
  \;\le\;
  C_{\mathrm{VK}}\, c \sum_{k\ge 0} 4^{-k} e^{-3\cdot2^{k}}
  \;=: VK\_B\_{\mathrm{budget}},
\]
independent of \(I\) and \(K\). This is the rigorous form of the \(VKWeightedSumHypothesis\) used in the main text.
\subsection*{B.7. VK weighted-sum hypothesis}

Combining Proposition~B.6 with the choice \(L = c / \log t_0\) gives
\[
  \sum_{k=0}^{K} 4^{-k} \nu_k(I)
  \;\le\;
  \sum_{k\ge 0} 4^{-k} \nu_k(I)
  \;\le\;
  C_{\mathrm{VK}}\, c \sum_{k\ge 0} 4^{-k} e^{-3\cdot2^{k}}
  \;=: VK\_B\_{\mathrm{budget}},
\]
independent of \(I\) and \(K\). This is the rigorous form of the \(VKWeightedSumHypothesis\) used in the main text.

\subsection*{B.8. Phase-velocity identity and neutralization}

\noindent\textbf{Theorem.} Let \(J(s) = \frac{\det_2(I-A(s))}{\mathcal O_\zeta(s)\, \zeta(s)}\), write \(\log J = U + iW\), and let
\[
  w(t) := \lim_{\varepsilon \downarrow 0} W\bigl(\tfrac12 + \varepsilon + it\bigr)
\]
exist in the sense of non-tangential boundary values. Then, as distributions on \(\mathbb R\),
\[
  -\,w'(t) \;=\; \pi\, \mu_{\mathrm{zeros}}(t) \;+\; \pi \sum_{\gamma} m_\gamma \,\delta_\gamma(t),
\]
where \(\mu_{\mathrm{zeros}}\) is the Poisson balayage of the off-critical zeros of \(\zeta\) and the atoms correspond to critical-line zeros \(\tfrac12 + i\gamma\) with multiplicity \(m_\gamma\).

\emph{Proof.} Fix a Whitney window \(\varphi_I\) of mean zero. For \(\varepsilon > 0\), define \(w_\varepsilon(t) := W(\tfrac12+\varepsilon+it)\). The Carleson bound \eqref{eq:carleson} implies that \(\{w_\varepsilon\}\) is bounded in VMO, hence \(J\) has no singular inner factor (cf.\ Garnett). Thus the inner part of \(J\) is a Blaschke product corresponding precisely to the zeros, and its derivative contributes the atomic measure \(\pi \sum m_\gamma \delta_\gamma\).

For the absolutely continuous part, note that the Cauchy--Riemann equations give
\[
  \int_{\mathbb R} \varphi_I(t)\, \bigl(-w_\varepsilon'(t)\bigr)\, dt
  =
  \iint_{Q(I)} \nabla U \cdot \nabla V_I^\varepsilon \, \sigma\, d\sigma\, dt,
\]
where \(V_I^\varepsilon\) is the Poisson extension of \(\varphi_I\) starting at height \(\varepsilon\). The integrand is dominated by
\[
  \|\nabla U\|_{L^2(Q(I), \sigma)} \cdot \|\nabla V_I^\varepsilon\|_{L^2(Q(I), \sigma)} \;\lesssim\; \sqrt{K_\xi}\, |I|^{1/2},
\]
uniformly in \(\varepsilon\). Letting \(\varepsilon \downarrow 0\) and using dominated convergence yields
\[
  \int_{\mathbb R} \varphi_I(t)\, \bigl(-w'(t)\bigr)\, dt
  =
  \pi \int_{\mathbb R} \varphi_I(t)\, d\mu_{\mathrm{zeros}}(t),
\]
because \(\nabla U\) represents the Poisson potential of the zero measure. Since such \(\varphi_I\) span a dense subset in the space of compactly supported test functions, the equality of distributions follows. \(\square\)

\subsection*{B.9. CR/Green pairing and outer cancellation}

\noindent\textbf{Lemma (Window energy bound).} Let \(\psi \in C_c^\infty([-2,2])\) be even with \(\int \psi = 1\), and define
\[
  \varphi_{I}(t) := \frac{1}{L} \psi\!\Bigl(\frac{t-t_0}{L}\Bigr)
\]
for a Whitney interval \(I\) of length \(L\). Let \(V_I\) be the harmonic extension of \(\varphi_I\) to \(\Omega\). Then there exists a constant \(C_{\mathrm{test}}\) depending only on \(\psi\) and the Whitney aperture such that
\[
  \iint_{Q(I)} |\nabla V_I|^2 \,\sigma\, d\sigma\, dt \;\le\; C_{\mathrm{test}}^2\, |I|.
\]

\emph{Proof.} Scale invariance reduces the statement to the reference interval \([-1,1]\). In that case \(\psi\) is fixed, so both \(\|\varphi\|_{L^2}\) and \(\|\partial_t \varphi\|_{L^2}\) are bounded. The harmonic extension can be written via the Poisson kernel
\[
  V(t,\sigma) = \int_{\mathbb R} \varphi(s) \, P_\sigma(t-s)\, ds,
\]
and standard computations show
\[
  \iint_{[-1,1]\times (0,1]} |\nabla V|^2 \sigma\, d\sigma\, dt
  \;\le\; C_\psi,
\]
for some \(C_\psi\) depending only on \(\psi\). Rescaling back to \(I\) multiplies the energy by \(|I|\), giving the desired bound with \(C_{\mathrm{test}}^2 = C_\psi\). \(\square\)

\noindent\textbf{Lemma (CR/Green identity).} For every Whitney interval \(I\),
\[
  \int_{\mathbb R} \varphi_I(t)\, \bigl(-w'(t)\bigr)\, dt
  =
  \iint_{Q(I)} \nabla U \cdot \nabla V_I \, \sigma\, d\sigma\, dt.
\]

\emph{Proof.} Since \(U\) and \(W\) are harmonic conjugates in \(\Omega \setminus Z(\xi)\), the divergence theorem applied to the vector field \(V_I \nabla U - U \nabla V_I\) over \(Q(I)\) (after removing small discs around zeros and letting their radii shrink to zero) yields the identity. The boundary term on the vertical sides collapses because \(V_I\) is supported slightly beyond \(I\), and the horizontal term equals the left-hand side after letting \(\sigma \to 0\). \(\square\)

\noindent\textbf{Lemma (Outer cancellation).} Write \(U = U_{\mathrm{zeros}} + U_{\mathrm{outer}}\) where \(U_{\mathrm{outer}} = \Re \log \mathcal O_\zeta - \Re \log \det_2(I-A)\). Then
\[
  \iint_{Q(I)} \nabla U_{\mathrm{outer}} \cdot \nabla V_I \, \sigma\, d\sigma\, dt = 0.
\]

\emph{Proof.} The function \(U_{\mathrm{outer}}\) is harmonic on \(\Omega\) with boundary values independent of \(J\)'s zeros (it comes from the outer normalization). Because \(\varphi_I\) has mean zero and \(V_I\) is its Poisson extension, the pairing \(\int_{\mathbb R} \varphi_I(t)\, \bigl(-w_{\mathrm{outer}}'(t)\bigr)\, dt\) vanishes. Applying the CR/Green identity to \(U_{\mathrm{outer}}\) and \(\varphi_I\) thus gives zero. \(\square\)

\noindent\textbf{Corollary (Quantitative CR/Green pairing).} Combining Lemmas B.9.1--B.9.3 yields
\[
  \Bigl|\int_{\mathbb R} \varphi_I(t)\, \bigl(-w'(t)\bigr)\, dt \Bigr|
  \le
  C_{\mathrm{test}}\, \sqrt{K_\xi}\, |I|^{1/2},
\]
as claimed in Section~5.

\section*{Appendix C. Implementation Notes}

\subsection*{C.1. Positive-index Abel summation (pseudocode)}

\begin{itemize}
  \item Goal: bound \(\sum_{n\le N} a(n) f(n)\) with \(A(k)=\sum_{m\le k} a(m)\).
  \item Identity: \(\sum_{n=1}^{N} a(n) f(n) = A(N) f(N) - \sum_{k=1}^{N-1} A(k)\,(f(k+1)-f(k))\).
  \item Apply with \(a(n)=n^{-it}\), \(f(n)=n^{-\sigma}\), and bound \(A(k)\) by Ford/Korobov.
  \item Convert \(\sum A(k)\,\Delta f\) to an integral using \(f'(x)=-\sigma x^{-(1+\sigma)}\).
\end{itemize}

\subsection*{C.2. Tactic strategy highlights (Lean)}

\begin{itemize}
  \item \emph{Range vs.\ Icc splitting:} use \texttt{sum\_union} and disjointness lemmas to move from \(\mathrm{range}\) sums to \(\mathrm{Icc}(1,N)\).
  \item \emph{Norm bounds:} \texttt{norm\_sum\_le}, \texttt{norm\_mul\_le}, and \texttt{gcongr} help propagate inequalities through sums and products.
  \item \emph{Carleson energy:} structure the proof as “weighted sum bound \(\Rightarrow\) energy bound” to avoid recomputation; keep constants symbolic.
  \item \emph{Wedge closure:} split into upper/lower estimates, then define \(\Upsilon\) and apply a single inequality lemma \((\Upsilon<1/2)\Rightarrow\) wedge.
\end{itemize}

\section*{Appendix D. Bibliography}

\begin{thebibliography}{99}
\bibitem{Ford2002}
K.~Ford, \emph{Vinogradov's integral and bounds for the Riemann zeta-function}, (2002).

\bibitem{Korobov1958}
N.~M. Korobov, \emph{Estimates of trigonometric sums and applications}, (1958).

\bibitem{dlVP1899}
C.~J. de la Vallée Poussin, \emph{Sur la fonction $\zeta(s)$ de Riemann et le nombre des nombres premiers inférieurs à une limite donnée}, (1899).

\bibitem{Titchmarsh}
E.~C. Titchmarsh, \emph{The Theory of the Riemann Zeta-Function}, Oxford Univ. Press.

\bibitem{Garnett}
J.~B. Garnett, \emph{Bounded Analytic Functions}, Springer.

\bibitem{SteinShakarchi}
E.~M. Stein and R.~Shakarchi, \emph{Real Analysis}, Princeton Univ. Press.

\bibitem{Axler}
S.~Axler, P.~Bourdon, W.~Ramey, \emph{Harmonic Function Theory}, Springer.
\end{thebibliography}

\end{document}
