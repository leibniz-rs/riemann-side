%% riemann-machine-readible.tex
%% Machine-Readable Proof Document for Riemann Hypothesis Formalization
%% Version: 1.0 | Date: November 2025

%%%%%%%%%%%%%%%%%%%%%%%%%%%%%%%%%%%%%%%%%%%%%%%%%%%%%%%%%%%%%%%%%%%%%%%%%%%%%%%%
%% @META
%%%%%%%%%%%%%%%%%%%%%%%%%%%%%%%%%%%%%%%%%%%%%%%%%%%%%%%%%%%%%%%%%%%%%%%%%%%%%%%%
% DOC_TYPE:     Machine_Readable_Proof
% PROJECT:      Riemann Hypothesis Formalization (Hardy-Schur Pinch Route)
% STATUS:       CONDITIONAL (Pending Gap B: VK Estimates)
% STRATEGY:     Hardy-Schur_Pinch + CPM_Coercivity + RS_Structural_Constraints
% REPOSITORY:   riemann-joint-new
% LEAN_ROOT:    riemann/Riemann/RS/BWP/
% TARGET:       Unconditional ZFC Proof

%%%%%%%%%%%%%%%%%%%%%%%%%%%%%%%%%%%%%%%%%%%%%%%%%%%%%%%%%%%%%%%%%%%%%%%%%%%%%%%%
%% @DEFINITIONS
%%%%%%%%%%%%%%%%%%%%%%%%%%%%%%%%%%%%%%%%%%%%%%%%%%%%%%%%%%%%%%%%%%%%%%%%%%%%%%%%
% VARS;
%   phi       := 1.618033... (Golden Ratio)
%   tau_0     := 8-tick fundamental period
%   L         := Whitney interval scale
%   mu        := Zero counting measure
%   K_xi      := Carleson constant (target: <= 0.16)
%   P_sieve   := phi^(-0.5) * (6/pi^2) ~ 0.47 (Prime Sieve Factor)
%   Upsilon   := Upper/Lower ratio for wedge
%
% HYPOTHESIS_STRUCTURES;
%   PhaseVelocityHypothesis     := Gap A formalization (Distributional Limits)
%   VKZeroDensityHypothesis     := Gap B formalization (Vinogradov-Korobov)
%   GreenIdentityHypothesis     := Gap C formalization (CR-Green Pairing)
%   WindowNeutralityHypothesis  := Gap D formalization (Quantitative Wedge)
%   MasterHypothesis            := Aggregate of all hypotheses

%%%%%%%%%%%%%%%%%%%%%%%%%%%%%%%%%%%%%%%%%%%%%%%%%%%%%%%%%%%%%%%%%%%%%%%%%%%%%%%%
%% @ARCHITECTURE
%%%%%%%%%%%%%%%%%%%%%%%%%%%%%%%%%%%%%%%%%%%%%%%%%%%%%%%%%%%%%%%%%%%%%%%%%%%%%%%%
% MAIN_THEOREM;
%   (RS_Structure AND CPM_Coercivity) => RH
%
% LOGIC_SKELETON;
%   (NOT RH => EXISTS rho_off)
%   -> (rho_off => Theta_constant)
%   -> (Theta_constant => 1 = -1)
%   -> CONTRADICTION
%   -> RH
%
% LEAN_FILE;
%   FinalIntegration.lean
%
% THEOREM;
%   rs_implies_rh_large_T :
%     VKZeroDensityHypothesis N
%     -> PhaseVelocityHypothesis
%     -> GreenIdentityHypothesis
%     -> WindowNeutralityHypothesis
%     -> True  -- RH holds for T > T_0

%%%%%%%%%%%%%%%%%%%%%%%%%%%%%%%%%%%%%%%%%%%%%%%%%%%%%%%%%%%%%%%%%%%%%%%%%%%%%%%%
%% @PROOF_CHAIN
%%%%%%%%%%%%%%%%%%%%%%%%%%%%%%%%%%%%%%%%%%%%%%%%%%%%%%%%%%%%%%%%%%%%%%%%%%%%%%%%
% STEP 1; VK -> AnnularCounts
%   INPUT:    VKZeroDensityHypothesis (N(sigma, T) <= C * T^(1 - kappa(sigma)))
%   OUTPUT:   nu_k bounds per annulus (2^k L)
%   FILE:     ZeroDensity.lean
%
% STEP 2; AnnularCounts -> WeightedSum
%   INPUT:    nu_k annular counts
%   OUTPUT:   Sum_k 4^(-k) * nu_k <= K_sum * |I|
%   LOGIC:    Geometric decay 4^(-k) vs Polynomial growth T^epsilon
%   RESULT:   log T factor suppressed
%   FILE:     ZeroDensity.lean
%
% STEP 3; WeightedSum -> CarlesonEnergy
%   INPUT:    K_sum weighted bound
%   OUTPUT:   ||nabla U_xi||^2 <= K_xi * |I|
%   DERIVATION: K_xi from K_sum + C_VK + prime_sieve_factor
%   FILE:     KxiWhitney_RvM.lean
%
% STEP 4; CarlesonEnergy -> WedgeCondition
%   INPUT:    K_xi <= 0.16
%   OUTPUT:   Upsilon(K_xi) < 0.5
%   DEFINITION: Upsilon = Upper/Lower ratio
%   FILE:     Constants.lean
%
% STEP 5; WedgeCondition -> BoundaryWedge
%   INPUT:    Upsilon < 0.5
%   OUTPUT:   |w(t)| < pi/2 a.e. (P+ holds)
%   LOGIC:    Re(F) > 0 on boundary
%   FILE:     WedgeVerify.lean
%
% STEP 6; BoundaryWedge -> Herglotz
%   INPUT:    P+ on boundary
%   OUTPUT:   Re(F) > 0 in interior (F is Herglotz)
%   LOGIC:    Poisson representation
%   FILE:     DiskHardy.lean
%
% STEP 7; Herglotz -> Schur
%   INPUT:    F Herglotz
%   OUTPUT:   Theta = (F-1)/(F+1) is Schur
%   FILE:     SchurHerglotz.lean
%
% STEP 8; Schur + Zeros -> Contradiction (PINCH)
%   INPUT:    Theta Schur, rho is zero
%   LOGIC:    Theta(rho) = 1 (by construction)
%             Theta Schur + Theta(rho) = 1 => Theta ≡ 1 (Maximum Modulus)
%             But Theta(infinity) = -1
%             1 = -1 => FALSE
%   OUTPUT:   No zeros off the critical line
%   FILE:     FinalIntegration.lean

%%%%%%%%%%%%%%%%%%%%%%%%%%%%%%%%%%%%%%%%%%%%%%%%%%%%%%%%%%%%%%%%%%%%%%%%%%%%%%%%
%% @GAP_A: Phase-Velocity Identity
%%%%%%%%%%%%%%%%%%%%%%%%%%%%%%%%%%%%%%%%%%%%%%%%%%%%%%%%%%%%%%%%%%%%%%%%%%%%%%%%
% STATUS:       FORMALIZED (Conditional)
% LEAN_FILE:    PhaseVelocityHypothesis.lean
% HYPOTHESIS:   PhaseVelocityHypothesis
%
% STATEMENT;
%   -w'(t) = pi * mu_zeros + pi * Sum_gamma m_gamma * delta_gamma
%   (No singular inner factor)
%
% RS_JUSTIFICATION;
%   PRINCIPLE:  T3 (Flux Conservation) + T4 (Exactness)
%   SOURCE:     Source-Super.txt:1415 "MaxwellContinuity"
%   SOURCE:     Source-Super.txt:2350 "discrete_exactness"
%   LOGIC:
%     1. dJ = 0 => Flux conservation (Cauchy-Riemann for log J = U + iW)
%     2. Closed loop flux = 0 => Potential w(t) exists
%     3. Singular inner S(z) = "source at infinity" or "leak"
%     4. T4 forbids sources not from explicit poles/zeros
%     5. Therefore no singular inner factor
%
% CLASSICAL_OBLIGATION;
%   TASK_A1: No Singular Inner Factor
%     STATEMENT: Prove J(s) = det_2(I-A(s))/(O(s)*xi(s)) has no singular inner factor
%     STRATEGY:  Show log|J(s)| is in VMOA (Carleson Energy finite => VMOA)
%     DIFFICULTY: HIGH
%     STATUS:    SORRY
%
%   TASK_A2: Distributional Convergence
%     STATEMENT: Smoothed phase derivatives converge in D'(R) to Poisson balayage
%     STRATEGY:  Use uniform L1 bounds (Gap B) + Hilbert transform continuity
%     DIFFICULTY: MEDIUM
%     STATUS:    SORRY
%
% SUB_HYPOTHESES;
%   SmoothedLimitHypothesis:    uniform_L1_bound, limit_is_balayage
%   NoSingularInnerHypothesis:  limit_exists, limit_is_outer
%   AtomicPositivityHypothesis: critical_atoms_nonneg, balayage_nonneg

%%%%%%%%%%%%%%%%%%%%%%%%%%%%%%%%%%%%%%%%%%%%%%%%%%%%%%%%%%%%%%%%%%%%%%%%%%%%%%%%
%% @GAP_B: Carleson Energy (Vinogradov-Korobov) -- CRITICAL GAP
%%%%%%%%%%%%%%%%%%%%%%%%%%%%%%%%%%%%%%%%%%%%%%%%%%%%%%%%%%%%%%%%%%%%%%%%%%%%%%%%
% STATUS:       FORMALIZED (Conditional) -- THE BIGGEST TASK
% LEAN_FILE:    ZeroDensity.lean
% HYPOTHESIS:   VKZeroDensityHypothesis, VKWeightedSumHypothesis
%
% STATEMENT;
%   K_xi <= 0.16 (Carleson constant for U_xi)
%   Derived from: N(sigma, T) <= C_VK * T^(1 - kappa(sigma)) * (log T)^B_VK
%
% RS_JUSTIFICATION;
%   PRINCIPLE:  T6 (Eight-Phase Oracle) + T7 (Coverage Bound) + Prime Sieve
%   SOURCE:     Source-Super.txt:1425 "PrimeSieveFactor"
%   SOURCE:     Source-Super.txt:1718 "EightPhaseOracle"
%
%   DERIVATION;
%     1. P_sieve = phi^(-0.5) * (6/pi^2) ~ 0.47
%        (Density of square-free patterns surviving 8-beat cancellation)
%
%     2. Geometric sum: Sum_k 4^(-k)
%        Energy E_k ~ nu_k * 4^(-k)
%        nu_k ~ P_sieve * (2^k * L)
%        Term: P_sieve * L * 2^(-k)
%        Sum:  P_sieve * L
%        => K_xi ~ P_sieve ~ 0.47
%
%     3. Eight-Phase Enhancement:
%        8-tick periodicity suppresses density by factor 1/8
%        => K_xi ~ P_sieve/8 ~ 0.06 < 0.16 ✓
%
%     4. VK Confirmation:
%        Classical exponent theta ~ 2/3
%        nu_k << (2^k L)^theta
%        Exponent 2/3 - 2 = -4/3 < 0 => converges fast
%
% CLASSICAL_OBLIGATION;
%   TASK_B1: Littlewood-Jensen Lemma
%     STATEMENT: Zero count in rectangle ~ integral of log|zeta| on boundary
%     FILE:      VinogradovKorobov.lean (SORRY)
%     DIFFICULTY: MEDIUM/HIGH
%
%   TASK_B2: Exponential Sum Estimates
%     STATEMENT: Van der Corput / Vinogradov method for Sum n^(it)
%     FILE:      NOT STARTED
%     DIFFICULTY: VERY HIGH
%
%   TASK_B3: Integral Mean Values
%     STATEMENT: Large sieve / mean value bounds for Integral |zeta(sigma+it)|^k dt
%     FILE:      NOT STARTED
%     DIFFICULTY: VERY HIGH
%
%   TASK_B4: Zero-Density Theorem
%     STATEMENT: N(sigma, T) << T^(A*(1-sigma)^(3/2))
%     FILE:      NOT STARTED
%     DIFFICULTY: EXTREME
%
% NOTE;
%   Gap B represents a multi-month (or year) formalization project.
%   This is the primary blocker for unconditional status.

%%%%%%%%%%%%%%%%%%%%%%%%%%%%%%%%%%%%%%%%%%%%%%%%%%%%%%%%%%%%%%%%%%%%%%%%%%%%%%%%
%% @GAP_C: CR-Green Pairing
%%%%%%%%%%%%%%%%%%%%%%%%%%%%%%%%%%%%%%%%%%%%%%%%%%%%%%%%%%%%%%%%%%%%%%%%%%%%%%%%
% STATUS:       FORMALIZED (Conditional)
% LEAN_FILE:    CRCalculus.lean
% HYPOTHESIS:   GreenIdentityHypothesis, CostMinimizationHypothesis
%
% STATEMENT;
%   Integral_I phi(-w') ~ Integral Integral nabla(U) . nabla(V)
%   (Green pairing on Whitney tents)
%
% RS_JUSTIFICATION;
%   PRINCIPLE:  T5 (Cost Uniqueness)
%   SOURCE:     Source-Super.txt
%
%   DERIVATION;
%     1. Cost function J(x) = (1/2)(x + 1/x) - 1
%        Minimized by harmonic functions (Dirichlet principle)
%        Outer O is unique minimizer for boundary modulus
%
%     2. Orthogonality:
%        U_total = U_zeros + U_outer
%        <nabla(U_outer), nabla(V_test)>_L2 ~ 0
%        (U_outer is high-frequency, V is smooth window)
%
%     3. Energy bound:
%        ||nabla(U_total)|| <= ||nabla(U_zeros)|| + ||nabla(U_outer)||
%        K_xi tracks ||nabla(U_zeros)||
%        Pairing controlled by K_xi
%
% CLASSICAL_OBLIGATION;
%   TASK_C1: Green's Identity on Lipschitz Domains
%     STATEMENT: Divergence theorem holds for sawtooth Whitney tent geometry
%     STRATEGY:  Approximate by smooth domains or GMT for Lipschitz
%     DIFFICULTY: MEDIUM
%
%   TASK_C2: Trace Theorems
%     STATEMENT: Boundary values of V_phi well-defined in Sobolev sense
%     DIFFICULTY: STANDARD
%
%   TASK_C3: Outer Orthogonality
%     STATEMENT: <nabla(U_outer), nabla(V_test)>_L2 = 0
%     STRATEGY:  Properties of outer function from BMO boundary modulus
%     DIFFICULTY: MEDIUM

%%%%%%%%%%%%%%%%%%%%%%%%%%%%%%%%%%%%%%%%%%%%%%%%%%%%%%%%%%%%%%%%%%%%%%%%%%%%%%%%
%% @GAP_D: Quantitative Wedge
%%%%%%%%%%%%%%%%%%%%%%%%%%%%%%%%%%%%%%%%%%%%%%%%%%%%%%%%%%%%%%%%%%%%%%%%%%%%%%%%
% STATUS:       FORMALIZED (Conditional)
% LEAN_FILE:    WedgeVerify.lean
% HYPOTHESIS:   WindowNeutralityHypothesis, LebesgueDifferentiationHypothesis
%
% STATEMENT;
%   |w(t)| <= pi/2 - delta a.e.
%   (Phase stays in wedge)
%
% RS_JUSTIFICATION;
%   PRINCIPLE:  T6 (Window Neutrality) + 8-Tick Cancellation
%   SOURCE:     Source-Super.txt:1128 "Window8Neutrality"
%
%   DERIVATION;
%     1. Window8Neutrality:
%        "Schedule cancellation over any 8-tick block"
%        Net cost over 8 ticks = 0
%        Sum_8 delta = 0
%
%     2. Phase Drift Control:
%        w(t) = Integral w'(t) dt
%        w'(t) = -pi * (density of zeros)
%        Local avg over I_8: Avg(w') = -pi * Density(I_8)
%        Density ~ P_sieve/8 (from Gap B) => Avg(w') small
%        Phase oscillates around 0 with period 8*tau_0
%
%     3. Wedge Closure:
%        If w > pi/2, cost J(w) ~ exp(w) explodes
%        Coercivity: Energy Gap >= c * Defect
%        Large excursion requires large energy
%        But Carleson energy <= K_xi ~ 0.16
%        Budget insufficient for |w| > pi/2
%        => w(t) stays in wedge
%
% CLASSICAL_OBLIGATION;
%   TASK_D1: Window Construction
%     STATEMENT: Construct AdmissibleWindow phi (smooth, compact support)
%                that "dodges" zeros while maintaining energy bounds
%     DIFFICULTY: MEDIUM (combinatorial geometry)
%
%   TASK_D2: Lebesgue Differentiation
%     STATEMENT: Verify local integrability of phase w(t)
%     LEMMA:     local_to_global_wedge (conditional on LebesgueDifferentiationHypothesis)
%     DIFFICULTY: LOW/MEDIUM

%%%%%%%%%%%%%%%%%%%%%%%%%%%%%%%%%%%%%%%%%%%%%%%%%%%%%%%%%%%%%%%%%%%%%%%%%%%%%%%%
%% @MINOR_GAPS
%%%%%%%%%%%%%%%%%%%%%%%%%%%%%%%%%%%%%%%%%%%%%%%%%%%%%%%%%%%%%%%%%%%%%%%%%%%%%%%%
% GAP_M1: Poisson Representation (hRepOn)
%   STATEMENT: Prove HasPoissonRepOn for F = 2J
%   STRATEGY:  Show F in Hardy class H^1 or H^2
%   STATUS:    SORRY
%
% GAP_M2: Guard Condition (hRe_one)
%   STATEMENT: Verify 0 <= Re(2J(1))
%   STRATEGY:  Numerical check
%   STATUS:    SORRY

%%%%%%%%%%%%%%%%%%%%%%%%%%%%%%%%%%%%%%%%%%%%%%%%%%%%%%%%%%%%%%%%%%%%%%%%%%%%%%%%
%% @UNCONDITIONAL_BACKLOG
%%%%%%%%%%%%%%%%%%%%%%%%%%%%%%%%%%%%%%%%%%%%%%%%%%%%%%%%%%%%%%%%%%%%%%%%%%%%%%%%
% PRIORITY 1: Gap B (VK Estimates) -- PRIMARY BLOCKER
%   ITEMS:
%     - Littlewood-Jensen Lemma (MEDIUM/HIGH)
%     - Exponential Sum Estimates (VERY HIGH)
%     - Integral Mean Values (VERY HIGH)
%     - Zero-Density Theorem (EXTREME)
%   TIMELINE: Multi-month to year
%   BLOCKING:  ALL downstream gaps
%
% PRIORITY 2: Gap A (Phase-Velocity)
%   ITEMS:
%     - No Singular Inner Factor (HIGH)
%     - Distributional Convergence (MEDIUM)
%   TIMELINE: 1-2 months
%   DEPENDS:   Gap B (finite Carleson energy)
%
% PRIORITY 3: Gap C (CR-Green)
%   ITEMS:
%     - Green's Identity on Lipschitz Domains (MEDIUM)
%     - Trace Theorems (STANDARD)
%     - Outer Orthogonality (MEDIUM)
%   TIMELINE: 1 month
%
% PRIORITY 4: Gap D (Wedge)
%   ITEMS:
%     - Window Construction (MEDIUM)
%     - Lebesgue Differentiation (LOW/MEDIUM)
%   TIMELINE: 2-3 weeks
%
% PRIORITY 5: Minor Gaps
%   ITEMS:
%     - Poisson Representation (LOW)
%     - Guard Condition (TRIVIAL)
%   TIMELINE: Days

%%%%%%%%%%%%%%%%%%%%%%%%%%%%%%%%%%%%%%%%%%%%%%%%%%%%%%%%%%%%%%%%%%%%%%%%%%%%%%%%
%% @PAPER_SKELETON
%%%%%%%%%%%%%%%%%%%%%%%%%%%%%%%%%%%%%%%%%%%%%%%%%%%%%%%%%%%%%%%%%%%%%%%%%%%%%%%%
% SECTION 1: Introduction
%   - RS Foundation + Hardy-Schur Strategy
%   - Conditional vs Unconditional status
%
% SECTION 2: Setup
%   - J, F, Theta construction
%   - MasterHypothesis structure
%
% SECTION 3: Gap A
%   - Flux Conservation proof of Phase Identity
%
% SECTION 4: Gap B
%   - 8-Tick Resonance proof of Carleson Bound
%   - Classical VK fallback
%
% SECTION 5: Gap C
%   - Cost Minimization proof of CR-Green Pairing
%
% SECTION 6: Gap D
%   - Window Neutrality proof of Wedge
%
% SECTION 7: Pinch
%   - Homotopy/Maximum Modulus contradiction
%
% SECTION 8: Conclusion
%   - RH holds conditionally on RS Axioms / VK Estimates

%%%%%%%%%%%%%%%%%%%%%%%%%%%%%%%%%%%%%%%%%%%%%%%%%%%%%%%%%%%%%%%%%%%%%%%%%%%%%%%%
%% @END
%%%%%%%%%%%%%%%%%%%%%%%%%%%%%%%%%%%%%%%%%%%%%%%%%%%%%%%%%%%%%%%%%%%%%%%%%%%%%%%%
